\chapter{Getting started}
The packet filtering has been with Linux kernel since version 1.1. It was first ported from BSD's “ipfw” and then enhanced in kernel 2.0 which by then controlled by “ipfwadm” tool. After then there was a rework which resulted the tool “ipchains” with many new features for kernel 2.2. The last attempt to rework has been done which then resulted “iptables” for kernel 2.4 and 2.6 that you are going to know about here.\newline

For the sake of simplicity (or God), they made it modular, so that you can use whatever module you want or you write to control packet filtering by Linux kernel. Modularity is the key to success for iptables which lasted from mid-1999 till now. So there is no need to rework again and again to result a tool which lasts just some years.\newline

There are many times you may use packet filtering, be it security, intrusion, hacking or whatever you call it, which disallows someone from accessing your data. Like it or not, your system is not secure and you should put all your efforts to protect your data if you want them and if they cost you more if they are lost or stolen. Undoubtedly, Linux packet filtering is powerful but no firewall on earth is really secure.\newline

As of any other firewall application, iptables is used to:
\begin{itemize}
\item filter packets based on their header information
\item pass traffic through (NAT / Masquerade / NAPT)
\item modify packet headers
\end{itemize}
and many more which will be discussed in next chapters.\newline

IPtables is just a tool that controls the “Netfilter” module which is loaded into Linux kernel by default in most distributions. Also it has the ability to load and control existing and new modules to control new targets and new matches which are not built-in. The Netfilter is a module that is loaded into kernel for easier implementation of the features of iptables. The iptables' package contains these modules and other executables to manipulate rules and configurations. The IP Connection Tracker module is used to track connections passed through iptables and the sessions in which these connections relate to. There are also many details which will be discussed in following chapters.\newline

This was really a quick introduction to iptables and I'm sure that I've said why I lessen the length of the book. I hope you more security than ever before.\newline
\section{Summary}
Well, you learnt a bit about the history of the iptables. You also learnt how the iptables generally works. The IP connection tracker and the Netfilter are a part of iptables' package. So, you had a bird's eye view on them in order to be introduced later. The general features of iptables' package has also been explored.